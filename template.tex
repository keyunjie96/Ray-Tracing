\documentclass[12pt, a4paper, UTF8]{article}
\usepackage{ctex}
\usepackage{amsmath}	
\usepackage{listings}
%\usepackage{verbtime}
\usepackage{lastpage}
\usepackage{apalike}
\usepackage{color,framed}
\usepackage{float}
\usepackage{graphics}
\usepackage{graphicx}
\usepackage{lastpage}%获得总页数
\usepackage{fancyhdr}
 \pagestyle{fancy}
 \cfoot{\thepage\ of \pageref{LastPage}}
 \renewcommand{\headrulewidth}{0.4pt}%改为0pt即可去掉页眉下面的横线
 \renewcommand{\footrulewidth}{0.4pt}%改为0pt即可去掉页脚上面的横线
\title{光线追踪实验报告}
\author{柯云劼\\计44 2014012086}
%\date{April 24, 2016}
\begin{document}
\maketitle
\section{基本功能}

\subsection{参数球与参数长方体的求交}
\subsubsection{球}
计算直线与参数球方程的交点,选取最近的一个。\par
这里可利用胡老师上课介绍的方法进行快速计算。\par
\subsubsection{长方体}
计算直线与六个面的交点。选取三组相对的面中交点最近的面,再从中选取最远的面。\par
检验约束情况,计算出该面的交点是否在长方体实际面的约束内;若是则认为与该长方体有交点,且为该交点。\par
\subsection{局部光照模型}
使用Phong光照明模型
\subsection{理想漫折射}
计算表面点P的漫反射光
$$I_d = I_pK_dcos(\theta)$$
其中$\theta$为从点P指向光源的向量和点P的法向量间的夹角
\subsection{镜面反射光}
计算表面点P的镜面反射光
$$I_s = I_pK_scos^n(\alpha)$$
其中$\alpha$为视线方向与反射方向的夹角

\subsection{光线追踪的实现}
根据物体表面的系数,先计算表明光照,然后根据反射折射定律计算发射光与折射光,递归地返回相应颜色值。\par
最后观测到的像素颜色即为局部光照与反射折射光的加权和。


\section{工程实现}
\subsection{读入}
读入config.txt文件来进行环境场景的配置和设置
\subsection{输出}
输出使用bmp类,将Color类矩阵的值保存为bmp文件。
\subsection{类结构}

Camera类:摄像头,管理最后返回的的像素信息,发射光线\par
RayTrace类:光线追踪,获得摄像头发射的光线与场景类进行交互,返回计算的值给摄像头\par
Scene类:场景,存储光源与物体等信息\par
Object类:物体的虚基类,存储交点和材质信息,并有读取、求交等纯虚函数;派生出Sphere、Box、Plane、Mesh、Model等类,其中Model类内置了k-d tree,方便求交运算\par
Inter类:交点,存储交点位置、距离和轴向等信息\par
Light类:光源的虚基类,存储光源位置,并有读取、计算颜色等纯虚函数;派生出PointLight、PlaneLight类\par

\section{拓展功能}
\subsection{抗锯齿效果(超采样)}
	\begin{figure}[h]
		\centering
		\includegraphics[width=\linewidth]{naive.png}
			\caption{无超采样}	
	\end{figure}
	\begin{figure}[h]
		\centering
		\includegraphics[width=\linewidth]{4sampling.png}
			\caption{16X超采样}	
	\end{figure}
实现:对于每个像素在领域内密集采样,将采样结果平均
\subsection{物体求交加速(k-d tree)}
对于复杂模型而言使用k-d tree能够有效加速求交运算,将每次$O(n^2)$的求交计算可降到$O(nlogn)$。
\subsection{复杂网格模型(obj模型读取)}
	\begin{figure}[h]
		\centering
		\includegraphics[width=\linewidth]{model_4sampling_shadow.png}
			\caption{复杂模型}	
	\end{figure}\newpage
读取obj格式文件,并生成数个面片物件,由模型物件统一管理,并构建k-d tree。
\subsection{多样化的材质效果}
可在配置文件中调整各参数比例,来实现多样化的材质效果,上图即实现了简单的玻璃球和大理石表面材质。
\subsection{贴图(UV映射、凹凸贴图)}
	\begin{figure}[h]
		\centering
		\includegraphics[width=\linewidth]{2shadow_2sampling_model_worldmap.png}
			\caption{贴图}
	\end{figure}\newpage	
如上所示,实现了UV映射,其中地球即是使用的UV映射。
\subsection{软阴影(面光源采样)}
	\begin{figure}[h]
		\centering
		\includegraphics[width=\linewidth]{naive}
			\caption{硬阴影}	
	\end{figure}
	\begin{figure}[h]
		\centering
		\includegraphics[width=\linewidth]{shadow}
			\caption{软阴影}	
	\end{figure}
实现:检测局部光照时,将面光源中一定数量的点视为点光源求阴影,最后将值平均即可
\par
\section{心得体会}
工程中遇到了许多需要扩展的效果,良好的OOP设计能使得后续迭代更顺利。
\bibliography{myrefs, hisrefs, herrefs}
\bibliographystyle{apalike}
\end{document}
